\setlength{\parskip}{0.3cm}%
\hyphenation{NoDaLiDa}

We are delighted to invite you to CG-FST NLP 2025, the ninth NoDaLiDa workshop on Constraint Grammar (CG). Constraint grammars often take text analysed with the help of finite state transducers (FST), which is why we have also invited papers on FST this year. The conference is being held as a physical event only, on March 5th 2025, at the NoDaLiDa conference in Tallinn.

Constraint Grammar is a grammar formalism developed in the Nordic countries. The main open source implementation FSTs, Helsinki finite state transducer, or HFST, was made in Finland. It is thus only to be expected that this workshop series is a pendant to NoDaLiDa/Baltic-HLT. The first CG workshop was arranged at NoDaLiDa in Odense in 2009, and it has been arranged at every NoDaLiDa conference ever since, except for at the 23rd NoDaLiDa in Reykjavik in 2021 (which was held mostly as a virtual event, due to the COVID pandemic).

Topic-wise, the papers in this year's workshop may be divided into three groups: Five of the papers deal mainly with one language, and present a spellchecker (Horváth, Rueter and Trosterud), grammar checker (Bick; Denbæk) or grammatical analysis (Gerstenberger; Trosterud and Vonen). Four of the papers take a more general approach, and deal with the functionalities of Constraint grammar (Swanson; Wiechetek and Unhammer) or FST (Pirinen and Moshagen) as proofing tools. Finally, one paper (Torstensson and Holmström) is a hybrid paper investigating the role of (possibly CG) annotated data may play for LLMs. This year's workshop covers, not only a wide range of applications, but also different languages (Esperanto, Romanian, Mansi, Tokelau, Greenlandic, Lule Sámi, Irish).

We would like to thank the members of the program committee (Eckhard Bick, Tino Didriksen, Kaili Müürisep, Daniel Glen Swanson and  Francis Tyers) for taking part in planning and organising the workshop. We would also like to thank the anonymous reviewers for reviewing the incoming papers. Without anonymous reviews there are no peer-reviewed proceedings, and their work is thus highly appreciated.


Trond Trosterud, General Chair

Linda Wiechetek and Flammie Pirinen, Program Co-Chairs
